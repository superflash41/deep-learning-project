\section{Planteamiento del Problema}
A pesar de los avances en el uso de aprendizaje profundo para la detección de incendios,
muchas de las arquitecturas existentes no son lo suficientemente ligeras para ser
desplegadas en un dron, lo que limita su aplicabilidad en entornos de monitoreo aéreo
en tiempo real. Aunque existen modelos optimizados para dispositivos con restricciones
computacionales, muchos de ellos no han sido actualizados con las versiones más recientes
de arquitecturas de detección de objetos, lo que podría afectar su precisión y eficiencia.

Este trabajo busca abordar esta problemática explorando modelos que logren un equilibrio
entre precisión y eficiencia computacional, permitiendo su implementación efectiva en drones
para la detección temprana de incendios forestales.